% --------------------------------------------------------------
% This is all preamble stuff that you don't have to worry about.
% Head down to where it says "Start here"
% --------------------------------------------------------------
 
\documentclass[10pt]{article}
 
\usepackage[margin=1in]{geometry} 
\usepackage{amsmath,amsthm,amssymb}
\usepackage{hyperref}
\usepackage{graphicx}
\usepackage{subfigure}
\usepackage{tabularx}
\usepackage{multirow}
\usepackage{float}
 
\newcommand{\N}{\mathbb{N}}
\newcommand{\Z}{\mathbb{Z}}
 
\newenvironment{theorem}[2][Theorem]{\begin{trivlist}
\item[\hskip \labelsep {\bfseries #1}\hskip \labelsep {\bfseries #2.}]}{\end{trivlist}}
\newenvironment{lemma}[2][Lemma]{\begin{trivlist}
\item[\hskip \labelsep {\bfseries #1}\hskip \labelsep {\bfseries #2.}]}{\end{trivlist}}
\newenvironment{exercise}[2][Exercise]{\begin{trivlist}
\item[\hskip \labelsep {\bfseries #1}\hskip \labelsep {\bfseries #2.}]}{\end{trivlist}}
\newenvironment{reflection}[2][Reflection]{\begin{trivlist}
\item[\hskip \labelsep {\bfseries #1}\hskip \labelsep {\bfseries #2.}]}{\end{trivlist}}
\newenvironment{proposition}[2][Proposition]{\begin{trivlist}
\item[\hskip \labelsep {\bfseries #1}\hskip \labelsep {\bfseries #2.}]}{\end{trivlist}}
\newenvironment{corollary}[2][Corollary]{\begin{trivlist}
\item[\hskip \labelsep {\bfseries #1}\hskip \labelsep {\bfseries #2.}]}{\end{trivlist}}
 
\begin{document}
 
% --------------------------------------------------------------
%                         Start here
% --------------------------------------------------------------
 
%\renewcommand{\qedsymbol}{\filledbox}
 
\title{Big Data Computing - $4^{th}$ Homework Report}
\author{Boem Davide, ID: 1176946, \texttt{davide.boem@studenti.unipd.it} \and Boscaro Nicola, ID: 1181356, \texttt{nicola.boscaro.1@studenti.unipd.it} \and Faccin Dario, ID: 1177736, \texttt{dario.faccin@studenti.unipd.it}}
\date{}
 
\maketitle

% $^{\circ}$

In this report we've briefly described the results obtained by running our code, on \textit{Cloud Veneto},that we've written for the $4^{th}$ homework.

\section{Results}

In Table \ref{tab:results} you can see what we have obtained running our code on \textit{Cloud Veneto}\footnote{We decided to round the average distances to 4 decimal places.}.

\begin{table}[H]
  \centering
  \begin{tabularx}{\textwidth}{c || p{1.5cm} | p{1.5cm} | c | c | p{1.7cm} | p{2.2cm} | p{1.5cm} | p{2cm} }
    & \textbf{Cores used by application} & \textbf{Cores used for each executor} & \textbf{numBlocks} & \textbf{k} & \textbf{Coreset construction (ms)} & \textbf{Computation final solution (ms)} & \textbf{Average distance} & \textbf{Dataset (Approximate size)}\\
\hline\hline
\textbf{1} & \centering 50 & \centering 4 & 12 & 30 &  &  &  & \multirow{11}{*}{\centering\texttt{500000}}\\
\textbf{2} & \centering 50 & \centering 4 & 12 & 20 &  &  &  & \\
\textbf{3} & \centering 20 & \centering 2 & 12 & 30 & \centering 74394 & \centering 101 & \centering 8.7761 & \\
\textbf{4} & \centering 20 & \centering 2 & 12 & 20 & \centering 216547 & \centering 85 & \centering 8.9701 & \\
\textbf{5} & \centering 20 & \centering 2 & 25 & 20 & \centering 20786 & \centering 136 & \centering 8.8788 & \\
\textbf{6} & \centering 20 & \centering 2 & 5 & 20 & \centering 21123 & \centering 39 & \centering 8.8486 & \\
\textbf{7} & \centering 10 & \centering 4 & 5 & 20 & \centering 17721 & \centering 22 & \centering 8.9782 & \\
\textbf{8} & \centering 10 & \centering 1 & 12 & 30 & \centering 19158 & \centering 234 & \centering 8.7899 & \\
\textbf{9} & \centering 10 & \centering 1 & 12 & 20 & \centering 29196 & \centering 50 & \centering 8.8988 & \\
\textbf{10} & \centering 10 & \centering 1 & 25 & 20 & \centering 55043 & \centering 127 & \centering 8.9163 & \\
\textbf{11} & \centering 10 & \centering 1 & 5 & 20 & \centering 28766 & \centering 22 & \centering 9.0405 & \\
\hline
\textbf{12} &  &  &  &  &  &  &  & \\

  \end{tabularx}
  \caption{Results obtained on \textit{Cloud Veneto}} \label{tab:results}
\end{table}

\section{Plots}

\section{Conclusions}
Da scrivere in inglese in una forma \textit{pi\`{u} meglio}:

In generale: l'average distance risulta a grandi linee costante tra le varie prove con lo stesso dataset. Eventuali eccezioni a quanto detto sotto sono prevalentemente nel tempo del Coreset Construction e ipotizzo siano dovute al "traffico" presente sul cloud (tempo che ci impiega a "iniziare" il programma).

Considerazioni su Cores used by application e Cores used for each executor:

Escludendo i risultati in cui si usano pi\`{u} cores used by application che risentono nel tempo di coreset construction a causa del "traffico" nel Cloud, si pu\`{o} notare, a parità di numBlocks e k, un maggior tempo impiegato nel Coreset Construction (come si poteva immaginare). Per quanto riguarda invece la Computation final solution il risultato è più variabile.

Considerazioni riguardo numBlocks e k:

Al diminuire di numBlocks diminuisce il tempo del Coreset construction e del Computation final solution. Diminuendo k invece aumenta il tempo del Coreset construction e diminuisce quello del Computation final solution.
%\begin{figure}[!h]
%\centering%
%\subfigure[\protect\url{road_and_sign.png}\label{src}]%
%{\includegraphics[scale = 0.35]{images/road_and_sign.eps}}\qquad\qquad
%\subfigure[\protect\url{src_after_canny.png}\label{srcCanny}]%
%{\includegraphics[scale = 0.35]{images/src_after_canny.eps}}\qquad\qquad
%\subfigure[\protect\url{src_after_hough.png}\label{srcHough}]%
%{\includegraphics[scale = 0.35]{images/src_after_hough.eps}}\qquad\qquad
%\subfigure[\protect\url{src_final.png}\label{final}]%
%{\includegraphics[scale = 0.35]{images/src_final.eps}}
%\caption{In this set of pictures we can see what the different methods in the program have done: \ref{src} is the source image, \ref{srcCanny} is the grayscale version of the source image in which I've applied the \textit{Canny()} method, \ref{srcHough} is the same image as \ref{srcCanny} where I've applied the methods \textit{HoughLines()} and \textit{HoughCircles()}, and in the end we have \ref{final} where there is what it's been asked by the assignment.\label{fig:imgresults}}
%\end{figure}

%\begin{thebibliography}{9}
%\bibitem{HoughCircle} 
%Hough Circle Transform,
%\\\texttt{https://docs.opencv.org/3.4.1/d4/d70/tutorial\_hough\_circle.html}
%\end{thebibliography}

%\begin{figure}[htbp]
%\centering
%\includegraphics[width=35mm]{eps/airplane.eps}% "%" necessario
%\qquad\qquad
%\includegraphics[width=45mm]{eps/lena.eps}
%\caption{Didascalia comune alle due figure}
%\end{figure}



% --------------------------------------------------------------
%     You don't have to mess with anything below this line.
% --------------------------------------------------------------
 
\end{document}