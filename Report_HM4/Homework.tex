% --------------------------------------------------------------
% This is all preamble stuff that you don't have to worry about.
% Head down to where it says "Start here"
% --------------------------------------------------------------
 
\documentclass[10pt]{article}
 
\usepackage[margin=1in]{geometry} 
\usepackage{amsmath,amsthm,amssymb}
\usepackage{hyperref}
\usepackage{graphicx}
\usepackage{subfigure}
\usepackage{tabularx}
\usepackage{multirow}
\usepackage{float}
 
\newcommand{\N}{\mathbb{N}}
\newcommand{\Z}{\mathbb{Z}}
 
\newenvironment{theorem}[2][Theorem]{\begin{trivlist}
\item[\hskip \labelsep {\bfseries #1}\hskip \labelsep {\bfseries #2.}]}{\end{trivlist}}
\newenvironment{lemma}[2][Lemma]{\begin{trivlist}
\item[\hskip \labelsep {\bfseries #1}\hskip \labelsep {\bfseries #2.}]}{\end{trivlist}}
\newenvironment{exercise}[2][Exercise]{\begin{trivlist}
\item[\hskip \labelsep {\bfseries #1}\hskip \labelsep {\bfseries #2.}]}{\end{trivlist}}
\newenvironment{reflection}[2][Reflection]{\begin{trivlist}
\item[\hskip \labelsep {\bfseries #1}\hskip \labelsep {\bfseries #2.}]}{\end{trivlist}}
\newenvironment{proposition}[2][Proposition]{\begin{trivlist}
\item[\hskip \labelsep {\bfseries #1}\hskip \labelsep {\bfseries #2.}]}{\end{trivlist}}
\newenvironment{corollary}[2][Corollary]{\begin{trivlist}
\item[\hskip \labelsep {\bfseries #1}\hskip \labelsep {\bfseries #2.}]}{\end{trivlist}}
 
\begin{document}
 
% --------------------------------------------------------------
%                         Start here
% --------------------------------------------------------------
 
%\renewcommand{\qedsymbol}{\filledbox}
 
\title{Big Data Computing - $4^{th}$ Homework Report}
\author{Boem Davide, ID: 1176946, \texttt{davide.boem@studenti.unipd.it} \and Boscaro Nicola, ID: 1181356, \texttt{nicola.boscaro.1@studenti.unipd.it} \and Faccin Dario, ID: 1177736, \texttt{dario.faccin@studenti.unipd.it}\footnote{Contact email}}
\date{}
 
\maketitle

% $^{\circ}$

In this report we've briefly described the results obtained by running our code, on \textit{Cloud Veneto},that we've written for the $4^{th}$ homework.

\section{Results}

In Table \ref{tab:results} you can see what we have obtained running our code on \textit{Cloud Veneto}\footnote{We decided to round the average distances to 4 decimal places.}.

\begin{table}[H]
  \centering
  \begin{tabularx}{\textwidth}{c || p{1.5cm} | p{1.5cm} | c | c | p{1.7cm} | p{2.2cm} | p{1.5cm} | p{2cm} }
    & \textbf{Cores used by application} & \textbf{Cores used for each executor} & \textbf{numBlocks} & \textbf{k} & \textbf{Coreset construction (ms)} & \textbf{Computation final solution (ms)} & \textbf{Average distance} & \textbf{Dataset (Approximate size)}\\
\hline\hline
\textbf{1} & \centering 50 & \centering 4 & 12 & 30 & \centering 59869 & \centering 128 & \centering 8.7856 & \multirow{11}{*}{\centering\texttt{500000}}\\
\textbf{2} & \centering 50 & \centering 4 & 12 & 20 & \centering 101217 & \centering 71 & \centering 8.8655 & \\
\textbf{3} & \centering 20 & \centering 2 & 12 & 30 & \centering 74394 & \centering 101 & \centering 8.7761 & \\
\textbf{4} & \centering 20 & \centering 2 & 12 & 20 & \centering 216547 & \centering 85 & \centering 8.9701 & \\
\textbf{5} & \centering 20 & \centering 2 & 25 & 20 & \centering 20786 & \centering 136 & \centering 8.8788 & \\
\textbf{6} & \centering 20 & \centering 2 & 5 & 20 & \centering 21123 & \centering 39 & \centering 8.8486 & \\
\textbf{7} & \centering 10 & \centering 4 & 5 & 20 & \centering 17721 & \centering 22 & \centering 8.9782 & \\
\textbf{8} & \centering 10 & \centering 1 & 12 & 30 & \centering 19158 & \centering 234 & \centering 8.7899 & \\
\textbf{9} & \centering 10 & \centering 1 & 12 & 20 & \centering 29196 & \centering 50 & \centering 8.8988 & \\
\textbf{10} & \centering 10 & \centering 1 & 25 & 20 & \centering 55043 & \centering 127 & \centering 8.9163 & \\
\textbf{11} & \centering 10 & \centering 1 & 5 & 20 & \centering 28766 & \centering 22 & \centering 9.0405 & \\
\hline
\textbf{12} & \centering 50 & \centering 4 & 12 & 30 & \centering 262678 & \centering 180 & \centering 9.8699 & \multirow{11}{*}{\centering\texttt{all}}\\
\textbf{13} & \centering 50 & \centering 4 & 12 & 20 & \centering 272945 & \centering 77 & \centering 10.0896 & \\
\textbf{14} & \centering 20 & \centering 2 & 12 & 30 & \centering 216521 & \centering 182 & \centering 9.8286 & \\
\textbf{15} & \centering 20 & \centering 2 & 12 & 20 & \centering 248079 & \centering 89 & \centering 10.1511 & \\
\textbf{16} & \centering 20 & \centering 2 & 25 & 20 & \centering 246411 & \centering 176 & \centering 10.0338 & \\
\textbf{17} & \centering 20 & \centering 2 & 5 & 20 & \centering 237019 & \centering 56 & \centering 10.0863 & \\
\textbf{18} & \centering 10 & \centering 4 & 5 & 20 & \centering 374144 & \centering 56 & \centering 10.1711 & \\
\textbf{19} & \centering 10 & \centering 1 & 12 & 30 & \centering 130110 & \centering 187 & \centering 9.7677 & \\ 
\textbf{20} & \centering 10 & \centering 1 & 12 & 20 & \centering 136337 & \centering 94 & \centering 9.9745 & \\ 
\textbf{21} & \centering 10 & \centering 1 & 25 & 20 & \centering 145994 & \centering 207 & \centering 10.1882 & \\ 
\textbf{22} & \centering 10 & \centering 1 & 5 & 20 & \centering 197308 & \centering 54 & \centering 10.1459 & \\ 
\hline
\textbf{23} & \centering - & \centering - & 12 & 30 & \centering 11021 & \centering 66 & \centering 8.6162 & \multirow{4}{*}{\centering\texttt{500000} on PC}\\
\textbf{24} & \centering - & \centering - & 12 & 20 & \centering 7079 & \centering 30 & \centering 8.7742 & \\
\textbf{25} & \centering - & \centering - & 25 & 20 & \centering 6780 & \centering 80 & \centering 8.7115 & \\
\textbf{26} & \centering - & \centering - & 5 & 20 & \centering 7012 & \centering 6 & \centering 8.7778 & \\

  \end{tabularx}
  \caption{Results obtained on \textit{Cloud Veneto}} \label{tab:results}
\end{table}

\section{Plots}

\section{Conclusions}

Osservando le tabelle con i risultati possiamo notare come mentre nella colonna Coreset Construction abbiamo dei tempi che oscillano in un certo intervallo, nel caso della colonna Computation of final solution abbiamo una proporzionalità diretta con l'aumentare di k o numBlocks.

Discorso a parte per l'Average Distance che parte alta e diminuisce fino a raggiungere un comportamento asintotico.

Per quanto riguarda invece il numero di core utilizzati abbiamo verificato che aumentando il parametro X (total number of cores used by the application) c'è un leggero miglioramento nel Coreset Construction, mentre aumentando il parametro Y non c'è stato alcun miglioramento in quanto a tempi (semmai un peggioramento in alcuni casi).

%\begin{figure}[!h]
%\centering%
%\subfigure[\protect\url{road_and_sign.png}\label{src}]%
%{\includegraphics[scale = 0.35]{images/road_and_sign.eps}}\qquad\qquad
%\subfigure[\protect\url{src_after_canny.png}\label{srcCanny}]%
%{\includegraphics[scale = 0.35]{images/src_after_canny.eps}}\qquad\qquad
%\subfigure[\protect\url{src_after_hough.png}\label{srcHough}]%
%{\includegraphics[scale = 0.35]{images/src_after_hough.eps}}\qquad\qquad
%\subfigure[\protect\url{src_final.png}\label{final}]%
%{\includegraphics[scale = 0.35]{images/src_final.eps}}
%\caption{In this set of pictures we can see what the different methods in the program have done: \ref{src} is the source image, \ref{srcCanny} is the grayscale version of the source image in which I've applied the \textit{Canny()} method, \ref{srcHough} is the same image as \ref{srcCanny} where I've applied the methods \textit{HoughLines()} and \textit{HoughCircles()}, and in the end we have \ref{final} where there is what it's been asked by the assignment.\label{fig:imgresults}}
%\end{figure}

%\begin{thebibliography}{9}
%\bibitem{HoughCircle} 
%Hough Circle Transform,
%\\\texttt{https://docs.opencv.org/3.4.1/d4/d70/tutorial\_hough\_circle.html}
%\end{thebibliography}

%\begin{figure}[htbp]
%\centering
%\includegraphics[width=35mm]{eps/airplane.eps}% "%" necessario
%\qquad\qquad
%\includegraphics[width=45mm]{eps/lena.eps}
%\caption{Didascalia comune alle due figure}
%\end{figure}



% --------------------------------------------------------------
%     You don't have to mess with anything below this line.
% --------------------------------------------------------------
 
\end{document}